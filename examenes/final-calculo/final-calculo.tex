\documentclass{article}

\usepackage{fullpage}
\usepackage{amsmath}
\usepackage{graphicx}
\usepackage[utf8]{inputenc}
\usepackage{pdfpages}
\usepackage{graphicx}
\usepackage{multicol}
\usepackage{tikz}
\usepackage{pgfplots}
\pgfplotsset{compat=1.12}


\title{Examen final de Cálculo Diferencial}
\author{M. en C. Reinaldo Arturo Zapata Peña}

\begin{document}
\includepdf[pages=-]{car-final-calc}

\section{Teoría} % (fold)
\label{sec:teoria}


\begin{figure}[b]
    \centering
  \begin{tikzpicture}
    \begin{axis}[
     x=60, y=30, clip=false, xmin=-0.30,xmax=2.2*pi, xlabel= $x$, ylabel=$f(x)$,
     ymin=-2.2,ymax=2.7, axis lines=middle, xtick={0,1.57,3.14,4.71,6.28},
     xticklabel style = {fill=white}, xticklabels={$0$,
     $\displaystyle\frac{\pi}{2}$,
     $\displaystyle\pi\,$,$\,\,\,\displaystyle\frac{3\pi}{2}$,$\,2\pi$}, set
     layers = axis on top ]
     \draw (axis cs:0,1) -- (axis cs:6.28,1);
     \draw (axis cs:0,2) -- (axis cs:6.28,2);
     \draw (axis cs:0,-1) -- (axis cs:6.28,-1);
     \draw (axis cs:0,-2) -- (axis cs:6.28,-2);
     \draw (axis cs:1.57,2) -- (axis cs:1.57,-2);
     \draw (axis cs:3.14,2) -- (axis cs:3.14,-2);
     \draw (axis cs:4.71,2) -- (axis cs:4.71,-2);
     \draw (axis cs:6.28,2) -- (axis cs:6.28,-2);
     \draw (axis cs:0.78,2) -- (axis cs:0.78,-2);
     \draw (axis cs:2.36,2) -- (axis cs:2.36,-2);
     \draw (axis cs:3.93,2) -- (axis cs:3.93,-2);
     \draw (axis cs:5.50,2) -- (axis cs:5.50,-2);
    \end{axis}
  \end{tikzpicture}
    \caption{especio  para graficar funciones.}
    \label{fig:graficar}
\end{figure}
\begin{enumerate}
\item Explique qué es una función par e impar.
\hfill \textbf{6 puntos.}

\item Haciendo uso de la figura \ref{fig:graficar}, haga un bosquejo de las
funciones $f(x)=\cos(2x)$ y $g(x)=2\sin(x)$ para el intervalo $0 \leq x \leq
2\pi$. Identifica cada una de ellas etiquet\'andolas.
\hfill \textbf{6 puntos.}

\item Explique con sus palabras que interpretación geométrica tiene la
derivada de una función.
\hfill \textbf{6 puntos.}

\item Si para una recta tangencial a una función $f(x)$ se conoce el punto
($a$,$f(a)$) y la pendiente de la recta tangente $m_{\text{tan}}$, escriba el
procedimiento para encontrar la recta perpendicular a la función en dicho punto.
\hfill \textbf{6 puntos.}

\item Explique con sus palabras que establece el teorema de l'H\^opital y las
situaciones en las que se puede utilizar.
\hfill \textbf{6 puntos.}

\end{enumerate}

% section teoría (end)



\section{Problemas} % (fold)
\label{sec:problemas}

\begin{enumerate}

\item Sea la función $f(x)= 2x^{2} -3x + 8$ encuentre las rectas tangente y
perpendicular a dicha función en el punto $x=4$. Indique además si el ángulo que
forma dicha recta respecto al eje positivo de las $x$ es mayor o menor a
$90^{\circ}$ y mayor o menor que $180^{\circ}$.

\item Dadas las siguientes funcines, 

\hfill \textbf{9 puntos.}
\begin{align}
f(x) &= \frac{x^{2} - 4x  + 4 }{x+5}, \label{eq:lim1}
\\
g(x) &= \frac{\sin^{2}(x)}{x^{2}-25}, \label{eq:lim2}
\end{align}
encuentre los puntos para los cuales estas funciones son indeterminadas.

\item Utilizando las funciones del inciso anterior, determine el límite para la
Ec. \eqref{eq:lim1} cuando $x=3$ y el límite para la Ec. \eqref{eq:lim2} cuando
$x\rightarrow2$.

\item Utilizando nuevamente las Ecs. \eqref{eq:lim1} y \eqref{eq:lim2} y la
regla de l'H\^opital, determine el límite de cada una de las funciones cuando
$x\rightarrow-5$.


\end{enumerate}






% section problemas (end)

\end{document}
