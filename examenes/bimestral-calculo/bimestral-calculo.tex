\documentclass[]{article}

\title{\textbf{\large EXAMEN BIMESTRAL}}
\date{}

% \usepackage{showframe}
\usepackage[spanish]{babel}
\decimalpoint
\usepackage{fullpage}
\usepackage{enumitem}
\usepackage{graphicx}
\usepackage[makeroom]{cancel}
\usepackage{multirow}
\usepackage{eso-pic}
\usepackage{amsmath}
\usepackage{multicol}
\usepackage{caption}
\usepackage{subcaption}
\usepackage{tikz}
\usepackage{pgfplots}
\pgfplotsset{compat=1.12}


\newcommand\BackgroundLogo{
\put(-150,290){
\parbox[b][\paperheight]{\paperwidth}{%
\vfill
\centering
\includegraphics[width=3cm]{/Users/reinaldo/Documents/clases/unitec/logo}%
\vfill
}}}


\begin{document}
\AddToShipoutPicture*{\BackgroundLogo}
\ClearShipoutPicture
\maketitle


\vspace{-5mm}{\scriptsize \textbf{S S -+}}

\vspace{15mm}
\noindent\fbox{
    \parbox{0.978\textwidth}{
        \begin{center}
            {\large  \textbf{EXAMEN BIMESTRAL}}
        \end{center}
    }}

\vspace{10mm}

\begin{tabular}{rcccccc}
\textbf{EXAMEN BIMESTRAL}  &  &  &  &  & \\
\textbf{DE: } & \multicolumn{3}{c}{ \qquad \qquad C\'alculo Diferencial \qquad \qquad} \\
\cline{2-4}  \cline{6-6}
 & \multicolumn{3}{c}{\qquad \qquad Nombre de la materia \qquad \qquad} & \qquad \qquad \quad &\qquad \textbf{Autoriz\'o}\qquad\, \\ \\ \\
\textbf{Nombre del Alumno:} \\ 
\cline{2-5} \\
\textbf{Grupo}: & \qquad \qquad  \qquad & & \textbf{Fecha:} & \\
\cline{2-3} \cline{5-5} \\ 
\textbf{Nombre del profesor:} & \multicolumn{4}{c}{M. en C. Reinaldo Arturo Zapata Pe\~na}  \\
\cline{2-5} 
\end{tabular}


\section*{\small Instrucciones} 
\vspace{-4mm}
\begin{enumerate}[nolistsep]
{\footnotesize
\item \textbf{Lee con atenci\'on todo el examen antes de resolverlo. Escribe todos los
datos que se solicitan con tinta negra o azul. Solo hay una respuesta correcta
por reactivo. No uses corrector, evita tachaduras, sobreponer letras y/o
n\'umeros; de lo contrario se anular\'a el reactivo.}

\item \textbf{El examen es un documento institucional por lo tanto no debes rayar,
dibujar o realizar cualquier otro escrito ajeno a los contenidos del examen o
que por instrucci\'on no se te hayan solicitado; de lo contrario se ANULARA el
examen.}

\item \textbf{No se permite hablar, voltear o pedir alg\'un material a compañeros y/o
profesor durante el examen. No sacar celular, aud\'ifonos o cualquier aparato
ajeno al examen; de no cumplir con lo especificado se ANULAR\'a el examen.}

\item \textbf{Si se sorprende a un alumno (os) copiando bajo cualquier forma o medio se
ANULA el examen.}

\item \textbf{Los ex\'amenes resueltos con l\'apiz no tienen derecho a revisi\'on o
aclaraciones.}

\item \textbf{Es importante anotar TODOS LOS PASOS o PROCEDIMIENTOS en todos los
problemas y que estos sean l\'ogicos y entendibles, no hacerlo anula la
respuesta, a\'un si esta es correcta, se deber\'a  remarcar el resultado con
tinta negra o azul.} }
\end{enumerate}
\vspace{-3mm}
\begin{center}
{\small \textbf{LA ANULACI\'ON DE EXAMEN EQUIVALE A CERO DE CALIFICACI\'ON.}}
\end{center}
% section Instrucciones (end)

\section*{Teor\'ia} % (fold)
\label{sec:teor'ia}

\begin{enumerate}
\item Explique con sus palabras que es una funci\'on, cual es
la variable dependiente y cual es la variable independiente.
\hfill \textbf{8 puntos.}


\item Explique qu\'e es una funci\'on par e impar.
\hfill \textbf{8 puntos.}

% \item Sea la funci\'on sinusoidal dada por
% \begin{equation*}
% f(x) = A \sin (\beta x + \varphi),
% \end{equation*}
% explique a qu\'e hacen referencia las variables  $A$, $\beta$, $x$ y $\varphi$.


\item Haciendo uso de la fugura \ref{fig:graficar}, haga un bosquejo de las
funciones $f(x)=\cos(x)$ y $g(x)=\sin(x)$ para el intervalo $0 \leq x \leq
2\pi$. Identifica cada una de ellas etiquet\'andolas.
\hfill \textbf{9 puntos.}

\item Complete los siguiente teoremas sobre l\'imites:
\hfill \textbf{9 puntos.}

Sean $f(x)$ y $g(x)$ dos funciones definidas y con l\'imites $L_{1}$ y $L_{2}$
en el punto $x=a$ de tal manera que 
\begin{align*}
\lim_{x \to a} f(x) = L_{1}, && \lim_{x \to a} g(x) = L_{2},
\end{align*}
entonces
\begin{align}
\lim_{x \to a} [f(x) + g(x)]     &=    \\ \nonumber \\
\lim_{x \to a} [f(x) \cdot g(x)] &=   \\ \nonumber \\
\lim_{x \to a} \left[\frac{f(x)}{g(x)}\right] &=  \qquad \qquad \qquad \forall \quad L_{2} \neq 
\end{align}
\setcounter{equation}{0}

\item Explique con sus palabras que interpretaci\'on geom\'etrica tiene la
derivada de una funci\'on.
\hfill \textbf{8 puntos.}

\item Escriba la ecuaci\'on de la recta y explique cada uno de sus elementos.
\hfill \textbf{8 puntos.}

\end{enumerate}

\begin{figure}[b]
    \centering
  \begin{tikzpicture}
    \begin{axis}[
     x=60,
     y=30,
     clip=false,
     xmin=-0.30,xmax=2.2*pi,
     xlabel= $x$,
     ylabel=$f(x)$,
     ymin=-2.2,ymax=2.7,
     axis lines=middle,
     xtick={0,1.57,3.14,4.71,6.28},
     xticklabel style = {fill=white},
     xticklabels={$0$, $\displaystyle\frac{\pi}{2}$, $\displaystyle\pi\,$,$\,\,\,\displaystyle\frac{3\pi}{2}$,$\,2\pi$},
     set layers = axis on top
     ]
     \draw (axis cs:0,1) -- (axis cs:6.28,1);
     \draw (axis cs:0,2) -- (axis cs:6.28,2);
     \draw (axis cs:0,-1) -- (axis cs:6.28,-1);
     \draw (axis cs:0,-2) -- (axis cs:6.28,-2);
     \draw (axis cs:1.57,2) -- (axis cs:1.57,-2);
     \draw (axis cs:3.14,2) -- (axis cs:3.14,-2);
     \draw (axis cs:4.71,2) -- (axis cs:4.71,-2);
     \draw (axis cs:6.28,2) -- (axis cs:6.28,-2);
     \draw (axis cs:0.78,2) -- (axis cs:0.78,-2);
     \draw (axis cs:2.36,2) -- (axis cs:2.36,-2);
     \draw (axis cs:3.93,2) -- (axis cs:3.93,-2);
     \draw (axis cs:5.50,2) -- (axis cs:5.50,-2);
    \end{axis}
  \end{tikzpicture}
    \caption{especio  para graficar funciones.}
    \label{fig:graficar}
\end{figure}
% section teor'ia (end)

\section*{Problemas} % (fold)
\label{sec:problemas}

\begin{enumerate}
\item Haciendo uso de la figura \ref{fig:graficar} trace la gr\'afia de la
funci\'on $f(x) = 2 \sin (3x)$ para el intervalo $0 \leq x \leq 2\pi$.
\hfill \textbf{8 puntos.}



\item Utilizando la definici\'on de l\'imite para la derivada de una funci\'on,
\hfill \textbf{8 puntos.}
\begin{equation*}
f'(x) = \lim_{\Delta x \to a} \frac{f(x + \Delta x) - f(x)}{\Delta x},
\end{equation*}
calcule la derivada de $f(x)= 3x^{2} - 6x + 2 $.

\item Determine la ecuaci\'on de la recta que pasa por el punto (-1,2) y que
tiene pendiente $m=3$. El \'angulo que forma esta recta con el eje de las $x$
?`es mayor o menor qeu $45^{\circ}$? ?`Es mayor o menor qeu $90^{\circ}$?

\hfill \textbf{8 puntos.}

\item Determine la ecuaci\'on de la recta que pasa por los puntos (4,5) y (2,3).
\hfill \textbf{8 puntos.}


\item Dadas las siguientes funcines, 
\hfill \textbf{9 puntos.}
\begin{align}
f(x) &= \frac{x^{2} - 4x  + 4 }{x+5}, \\ \nonumber \\
g(x) &= \frac{\sin^{2}(x)}{x^{2}-25},
\end{align}
\setcounter{equation}{0}
encuentre los puntos para los cuales estas funciones son indeterminadas.
\item Determine  
\begin{equation*}
\lim_{x \to 0} \frac{\sin(x)}{x},
\end{equation*} 
calculando el l\'imite por la izquierda ($0^{-}$) derecha
($0^{+}$)

\item Calcule la derivada de las siguientes funciones:
\hfill \textbf{9 puntos.}
\begin{align}
f(x) &= 8x^{4} - 6x^{3} + 12x^{2} -3x +6, \\ \nonumber \\
g(x) &= 3x^{2}\cos(x), \\ \nonumber \\
h(x) &= \frac{\sin(x)}{2x^2}, \\ \nonumber 
\end{align}
\end{enumerate}
\setcounter{equation}{0}

% section problemas (end)

\newpage

\begin{center}
{\sc \huge Respuestas}
\end{center}

\section*{Teor\'ia} % (fold)
\label{sec:teoria}

\begin{enumerate}

\item Es una relaci\'on entre un conjunto dado $x$, llamado dominio, y otro
conjunto de elementos $y$, llamado codominio, de forma que a cada elemento del
dominio le corresponde un \'unico elemento del codominio. La variable
dependiente, generalmente $x$, es aquella a la que puede adquirir cualquier
valor que se encuentra en el dominio y la variable dependiente, generalmente
$y$, es aquella que adquiere valores en el contradominio y depende de que valor
se le haya asignado a la variable independiente.

\item Funci\'on par: $f(-x) = f(x)$.
Funci\'on impar: $f(-x) = -f(x)$.

\item Gr\'afica: $f(x)=\sin(x)$; $g(x)=\cos(x)$.
\begin{figure}[h!]
    \centering
  \begin{tikzpicture}
    \begin{axis}[
     x=60,
     y=30,
     clip=false,
     xmin=-0.30,xmax=2.2*pi,
     xlabel= $x$,
     ylabel=$f(x)$,
     ymin=-2.2,ymax=2.7,
     axis lines=middle,
     xtick={0,1.57,3.14,4.71,6.28},
     xticklabel style = {fill=white},
     xticklabels={$0$, $\displaystyle\frac{\pi}{2}$, $\displaystyle\pi\,$,$\,\,\,\displaystyle\frac{3\pi}{2}$,$\,\pi$},
     set layers = axis on top
     ]
     \draw (axis cs:0,1) -- (axis cs:6.28,1);
     \draw (axis cs:0,2) -- (axis cs:6.28,2);
     \draw (axis cs:0,-1) -- (axis cs:6.28,-1);
     \draw (axis cs:0,-2) -- (axis cs:6.28,-2);
     \draw (axis cs:1.57,2) -- (axis cs:1.57,-2);
     \draw (axis cs:3.14,2) -- (axis cs:3.14,-2);
     \draw (axis cs:4.71,2) -- (axis cs:4.71,-2);
     \draw (axis cs:6.28,2) -- (axis cs:6.28,-2);
     \draw (axis cs:0.785,2) -- (axis cs:0.785,-2);
     \draw (axis cs:2.36,2) -- (axis cs:2.36,-2);
     \draw (axis cs:3.93,2) -- (axis cs:3.93,-2);
     \draw (axis cs:5.497,2) -- (axis cs:5.497,-2);
    \addplot[domain=0:2*pi,samples=200,red]{sin(deg(x))}
                                node[right,pos=0.9,font=\footnotesize]{$\sin x$};
    \addplot[domain=0:2*pi,samples=200,blue]{cos(deg(x))}
                                node[right,pos=1,font=\footnotesize]{$\cos x$};
    \end{axis}
  \end{tikzpicture}
    % \caption{especio  para graficar funciones.}
    \label{fig:graficar2}
\end{figure}

\item Teoremas:
\begin{align}
\lim_{x \to a} [f(x) + g(x)]     &=  L_{1} + L_{2}  \\ \nonumber \\
\lim_{x \to a} [f(x) \cdot g(x)] &= L_{1} \cdot L_{2} \\ \nonumber \\
\lim_{x \to a} \left[\frac{f(x)}{g(x)}\right] &= \frac{L_{1}}{L_{2}} \qquad 
\forall \quad L_{2} \neq 0
\end{align}
\setcounter{equation}{0}

\item La derivada de una funci\'on evaluada en un punto nos proporciona el valor
de la pendiente de la recta tangente a la funci\'on en dicho punto.


\item Ecuaci\'on: $y = mx + b$

$y$: variable dependiente,

$x$: variable independiente,

$m$: pendiente,

$b$: ordenada al origen o instersecci\'on de la recta con el eje $y$.
\end{enumerate}
\newpage

\section*{Problemas} % (fold)
\label{sec:problemas2}

\begin{enumerate}
    
\item Gr\'afica: $2\sin(3x)$.
\begin{figure}[h!]
    \centering
  \begin{tikzpicture}
    \begin{axis}[
     x=60,
     y=30,
     clip=false,
     xmin=-0.30,xmax=2.2*pi,
     xlabel= $x$,
     ylabel=$f(x)$,
     ymin=-2.2,ymax=2.7,
     axis lines=middle,
     xtick={0,1.57,3.14,4.71,6.28},
     xticklabel style = {fill=white},
     xticklabels={$0$, $\displaystyle\frac{\pi}{2}$, 
     $\displaystyle\pi\,$,$\,\,\,\displaystyle\frac{3\pi}{2}$,$\,\pi$},
     set layers = axis on top
     ]
     \draw (axis cs:0,1) -- (axis cs:6.28,1);
     \draw (axis cs:0,2) -- (axis cs:6.28,2);
     \draw (axis cs:0,-1) -- (axis cs:6.28,-1);
     \draw (axis cs:0,-2) -- (axis cs:6.28,-2);
     \draw (axis cs:1.57,2) -- (axis cs:1.57,-2);
     \draw (axis cs:3.14,2) -- (axis cs:3.14,-2);
     \draw (axis cs:4.71,2) -- (axis cs:4.71,-2);
     \draw (axis cs:6.28,2) -- (axis cs:6.28,-2);
     \draw (axis cs:0.785,2) -- (axis cs:0.785,-2);
     \draw (axis cs:2.36,2) -- (axis cs:2.36,-2);
     \draw (axis cs:3.93,2) -- (axis cs:3.93,-2);
     \draw (axis cs:5.497,2) -- (axis cs:5.497,-2);
    \addplot[domain=0:2*pi,samples=200,red]{2*sin(deg(3*x))}
                                node[left,pos=.9,font=\footnotesize]{$2\sin(3x)$};
    \end{axis}
  \end{tikzpicture}
    \label{fig:graficar3}
\end{figure}

\item $f(x)= 3x^{2} - 6x + 2 $
\begin{align*}
f'(x) 
&= \lim_{\Delta x \to a} 
\frac{3(x + \Delta x)^{2} - 6(x + \Delta x) + 2 - 3x^{2} + 6x - 2}
{\Delta x} \\ \\
&=  \lim_{\Delta x \to a} 
\frac{3(x^{2} + 2x\Delta x + \Delta x^{2}) - 6(x + \Delta x) + 2 - 3x^{2} + 6x - 2}
{\Delta x} \\ \\
&=  \lim_{\Delta x \to a} 
\frac{6x\Delta x + 3 \Delta x^{2} -6\Delta x}{\Delta x} \\ \\
&=  \lim_{\Delta x \to a} 
6x + 3\Delta x -6 \\ \\
&= 6x - 6.
\end{align*}

\item Ecuaci\'on de la recta:
\begin{align*}
y - y_{1} &= m (x - x_{1}) \\
y - 2 &= 3(x - (-1)) \\
y &= 3x +3 +2 \\
y &= 3x +5
\end{align*}
El \'angulo que forma respecto al eje $x$ es mayor a $45^{\circ}$ y menor que
$90^{\circ}$.

\item Ecuaci\'on d la recta:
\begin{equation*}
m = \frac{y_{2} - y_{1}}{x_{2}-x_{1}} = \frac{ 3 - 5}{2 - 4} = \frac{-2}{-2} = 1 \\ \\
\end{equation*}
\begin{align*}
y - y_{1} &= m (x - x_{1}) \\
y - 5 &= (1) (x - 4) \\
y &= x - 4 +5 \\
y &= x+1
\end{align*}

\item Indeterminaciones:
\begin{align}
x+5 &= 0 \nonumber \\  
x&=-5 \\ \nonumber \\
x^{2}-25 &= 0 \nonumber \\
x &= \sqrt{25} \nonumber\\
x &= \pm 5
\end{align}
\setcounter{equation}{0}

\item L\'imite:
\begin{equation*}
\lim_{x \to 0} \frac{\sin(x)}{x} = 1,
\end{equation*} 

\item Derivadas:
\begin{align}
% f(x) &= 8x^{4} - 6x^{3} + 12x^{2} -3x +6, \\ \nonumber \\
f'(x) &= 32x^{3} - 18x^{2} + 24x -3, \\ \nonumber \\
% g(x) &= 3x^{2}\cos(x), \\ \nonumber \\
g'(x) &= -3x^{2}\sin(x) + 6x\cos(x) , \\ \nonumber \\
% h(x) &= \frac{\sin(x)}{2x^2}, \\ \nonumber \\
h'(x) &= \frac{\cos(x)}{2x^2} - \frac{\sin(x)}{x^{3}}, \\ \nonumber
\end{align}
\setcounter{equation}{0}


\end{enumerate}



\end{document}

